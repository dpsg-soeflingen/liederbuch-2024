\documentclass[11 pt, a5paper]{article}

\usepackage[left=1cm,right=1cm,top=2cm,bottom=2cm]{geometry}
\usepackage{calc}

% Use sans-serif font.
\renewcommand{\familydefault}{\sfdefault}

% General Chord Definition.
\newcommand{\Ch}[2]{%
    \makebox[0pt][c]{\hspace{\widthof{#2}}\raisebox{0.9em}[1.7em]{\textbf{#1}}}%
    #2%
}
% For brevity of notation, all chords as simple commands.
\newcommand{\Am}[1]{\Ch{Am}{#1}}
\newcommand{\E}[1]{\Ch{E}{#1}}

% The song-environment.
\newcommand{\SongTitle}[2]{
    {\centering\mbox{\textbf{\large#1}}\par}
    {\centering\mbox{#2}\par}
    \bigskip
}
\newenvironment{song}[2]{
    \SongTitle{#1}{#2}
}{\newpage}

% The chorus-environment indents the chorus.
\newenvironment{chorus}[0]{
    \bigskip
    \list{}{
        \itemindent\parindent
        \leftmargin 3em
    }
    \item\relax
    \textbf{Refrain: } \\
}{
    \endlist
    \bigskip
}

\setlength\parindent{0pt}

\begin{document}

% Each song is in its own environment.
% I would suggest to separate songs into their own .tex-files
% and include them into the entire book-tex-file.
\begin{song}{Hello there}{Obi-Wan Kenobi and Friends}
    % Each chord is centered over the given text-snippet.
    Das ist ein \Am{toller} Songtext. \\
    Und noch eine super \Am{Zeile}! \\
    Und jetzt eine richtig lange Zeile um zu schauen wie das aussieht! \\
    Uuuuhhhh!! Schick!
    %
    \begin{chorus}
    Es tut mir leid Pocahontas! \\
    Ich hoffe du weisst das! \\
    Tut mir so \E{leid} Pocahontas! \\
    POCAHONTAS!
    \end{chorus}
    %
    Hello there.
\end{song}

\begin{song}{Deine Schuld}{Die Aerzte}
    Hast du dich heute schon geaegert? War es heute wieder schlimm? \\
    Hast du dich wieder gefragt warum kein Mensch was unternimmt? \\
\end{song}

\end{document}
